\documentclass{jsarticle}

 \usepackage{ascmac}
 \usepackage[dvipdfmx]{graphicx}
 \usepackage[dvipdfmx]{color}
 \usepackage{amssymb,amsmath,amsthm}
 \usepackage{graphics}
 \usepackage{fancybox, tcolorbox}
 \tcbuselibrary{raster,skins, breakable}
 \usepackage{nccmath}
 \usepackage{tikz}
 \usetikzlibrary{intersections, calc, cd}
 \usepackage{bm}
 \usepackage[italicdiff]{physics}
 \usepackage{titlesec}
 \usepackage{mathtools}
 \usepackage{enumerate}
 \usepackage{wrapfig}

 \newcommand{\myproof}[1]{
  \begin{tcolorbox}[
    empty,top = -2pt, breakable = true,
    underlay = {
      \draw[line width = 5pt, color = teal] (frame.north west) -- (frame.south west);
      },
    underlay unbroken = {
      \fill (frame.south east) -- ([xshift = -5pt]frame.south east) --([xshift = -5pt, yshift = -7pt]frame.south east) -- ([yshift = -7pt]frame.south east) -- cycle;
    },
    underlay last = {
        \fill (frame.south east) -- ([xshift = -5pt]frame.south east) --([xshift = -5pt, yshift = -7pt]frame.south east) -- ([yshift = -7pt]frame.south east) -- cycle;
      }
    ]
    {\it Proof.}

    {#1}
  \end{tcolorbox}
 }

 \numberwithin{equation}{section}
 \setcounter{tocdepth}{3}

\theoremstyle{definition}
\newtheorem{dfn}{定義}[section]
\newtheorem{exa}{例}[section]
\newtheorem{thm}[dfn]{定理}
\newtheorem{prop}[dfn]{命題}
\newtheorem{note}[dfn]{注意}
\newtheorem{prob}[dfn]{問}
\newtheorem{coro}[dfn]{系}


\title{教科書67pの17.54〜17.57の説明}
\author{suzuka-0602}
\date{\today}

% \usepackage{fancyhdr}
% \pagestyle{fancy}
%     \lfoot{}
%     \cfoot{\thepage}
%     \rfoot{}

\begin{document}

\maketitle
% \tableofcontents
% \clearpage

今回の文章について,内容理解しながら書いているため,間違えなどがあると思われます.
そのときは何かしらでdmしていただくか,somaaruaru@gmail.comまで連絡してください
このpdfはgithub上で見れるようにしています.元のコードもgithub上にあるため,気になる方は確認してみてください.



\begin{figure}[h]
   \centering
\tikzset{every picture/.style={line width=0.75pt}} %set default line width to 0.75pt        

\begin{tikzpicture}[x=0.75pt,y=0.75pt,yscale=-0.7,xscale=0.85]
%uncomment if require: \path (0,300); %set diagram left start at 0, and has height of 300

%Shape: Rectangle [id:dp7472215261731835] 
\draw   (130.1,42.9) -- (427.35,42.9) -- (427.35,283.83) -- (130.1,283.83) -- cycle ;
%Straight Lines [id:da2394938771272458] 
\draw    (130.1,283.83) -- (130.1,44.9) ;
\draw [shift={(130.1,42.9)}, rotate = 90] [color={rgb, 255:red, 0; green, 0; blue, 0 }  ][line width=0.75]    (10.93,-3.29) .. controls (6.95,-1.4) and (3.31,-0.3) .. (0,0) .. controls (3.31,0.3) and (6.95,1.4) .. (10.93,3.29)   ;
%Straight Lines [id:da8069617214090359] 
\draw    (130.1,283.83) -- (425.35,283.83) ;
\draw [shift={(427.35,283.83)}, rotate = 180] [color={rgb, 255:red, 0; green, 0; blue, 0 }  ][line width=0.75]    (10.93,-3.29) .. controls (6.95,-1.4) and (3.31,-0.3) .. (0,0) .. controls (3.31,0.3) and (6.95,1.4) .. (10.93,3.29)   ;
%Curve Lines [id:da9662565065194997] 
\draw    (130.1,283.83) .. controls (184.47,249.48) and (174.32,261.97) .. (227.25,200.29) ;
%Curve Lines [id:da8328073699059954] 
\draw    (227.25,200.29) .. controls (304.1,189.36) and (304.82,189.36) .. (364.27,150.32) ;
%Straight Lines [id:da1501075516279594] 
\draw    (227.25,200.29) -- (250.45,43.36) ;
%Straight Lines [id:da6473125650127264] 
\draw    (204.05,192.48) -- (225.35,199.65) ;
\draw [shift={(227.25,200.29)}, rotate = 198.6] [color={rgb, 255:red, 0; green, 0; blue, 0 }  ][line width=0.75]    (10.93,-3.29) .. controls (6.95,-1.4) and (3.31,-0.3) .. (0,0) .. controls (3.31,0.3) and (6.95,1.4) .. (10.93,3.29)   ;

% Text Node
\draw (439.74,278.17) node [anchor=north west][inner sep=0.75pt]   [align=left] {{\fontfamily{ptm}\selectfont T}};
% Text Node
\draw (125.1,20) node [anchor=north west][inner sep=0.75pt]   [align=left] {{\fontfamily{ptm}\selectfont P}};
% Text Node
\draw (164.62,135.3) node [anchor=north west][inner sep=0.75pt]   [align=left] {{\fontfamily{ptm}\selectfont 固体}};
% Text Node
\draw (284.25,116.56) node [anchor=north west][inner sep=0.75pt]   [align=left] {{\fontfamily{ptm}\selectfont 液体}};
% Text Node
\draw (274.82,223.52) node [anchor=north west][inner sep=0.75pt]   [align=left] {{\fontfamily{ptm}\selectfont 気体}};
% Text Node
\draw (171,177.5) node [anchor=north west][inner sep=0.75pt]   [align=left] {{\fontfamily{ptm}\selectfont {\scriptsize 三重点}}};


\end{tikzpicture}
\caption{\(T\)と\(P\)での相図}
\end{figure}
\section{導入}
まず,ルールとして,\(D(x,y)\)を用いてある引数xとyによってそのx,yによる変化の自由度を関数\(D\)と表記する
.\\
2つの相での平衡において,$D(T, P) = 1$となるのは図を見て実感できるだろう. $D(T, v) = D(u, v) = 2$と表せられる.
(ここでは\(uとv\)はそれぞれ粒子数NでUとVを割ったものとする)



\end{document}